\documentclass{article}
\usepackage[utf8]{inputenc}
\usepackage[version=4]{mhchem}
\usepackage{chemfig}
\usepackage{amsmath}
\usepackage{geometry}
\geometry{margin=1in}

\title{Chemistry sample 2 question solutions for term final}
\author{@aranyaadheu}
\date{24 Dec, 2025}

\begin{document}

\maketitle

\section*{1. Covalent and Ionic Compounds}
\textbf{Definition:}
\begin{itemize}
    \item \textbf{Ionic Compounds:} Formed by the complete transfer of electrons from a metal to a non-metal, resulting in electrostatic attraction between cations and anions.
    \item \textbf{Covalent Compounds:} Formed by the sharing of electron pairs between two non-metal atoms.
\end{itemize}

\textbf{Comparison of Properties:}
\begin{center}
\begin{tabular}{|l|l|l|}
\hline
\textbf{Property} & \textbf{Ionic Compounds} & \textbf{Covalent Compounds} \\ \hline
Bonding Type & Electrostatic attraction & Electron pair sharing \\ \hline
Physical State & Usually crystalline solids & Solids, liquids, or gases \\ \hline
Melting/Boiling Point & Very high & Relatively low \\ \hline
Conductivity & Conducts in molten/aqueous state & Generally non-conductors \\ \hline
Solubility & Soluble in polar solvents (water) & Soluble in non-polar solvents \\ \hline
\end{tabular}
\end{center}

\section*{2. Resonance Structures and Formal Charges}
\textbf{Sulfate Ion (\ce{SO4^{2-}}):}
The most stable resonance structure involves two $S=O$ double bonds and two $S-O$ single bonds to minimize formal charges.
\[ \text{Formal Charge (FC)} = \text{Valence electrons} - (\text{Non-bonding electrons} + \frac{1}{2}\text{Bonding electrons}) \]
For $S=O$: $FC = 6 - (0 + 4) = 0$. For $S-O^-$: $FC = 6 - (6 + 1) = -1$. Sulfur: $6 - (0 + 6) = 0$.

\textbf{Carbonate Ion (\ce{CO3^{2-}}):}
Carbonate has three equivalent resonance structures where one oxygen is double-bonded and two are single-bonded with a $-1$ charge each.

\section*{3. Factors Governing Bond Formation}
\begin{itemize}
    \item \textbf{Ionic Bond:}
    \begin{enumerate}
        \item Low Ionization Energy of the metal.
        \item High Electron Affinity of the non-metal.
        \item High Lattice Energy of the resulting compound.
    \end{enumerate}
    \item \textbf{Covalent Bond:}
    \begin{enumerate}
        \item High electronegativity of both atoms.
        \item Small difference in electronegativity ($\Delta EN < 1.7$).
        \item High electron density between nuclei.
    \end{enumerate}
\end{itemize}

\section*{4. Lattice Energy}
\textbf{Definition:} Lattice energy is the energy released when one mole of an ionic crystalline compound is formed from gaseous ions.
\[ \ce{Na+(g) + Cl-(g) -> NaCl(s)} + \text{Lattice Energy} \]
\textbf{Example:} For \ce{NaCl}, the lattice energy is approximately $-787 \text{ kJ/mol}$. The negative sign indicates energy is released, making the crystal stable.

\section*{5. Trend in Lattice Energy: NaCl, KCl, RbCl}
Lattice energy ($U$) is governed by Coulomb's Law: $U \propto \frac{Q_1 Q_2}{r_0}$, where $r_0$ is the sum of ionic radii.
\begin{itemize}
    \item All three have the same charges ($+1, -1$).
    \item Ionic radii increase in the order: $\ce{Na+} < \ce{K+} < \ce{Rb+}$.
    \item Therefore, the inter-ionic distance $r_0$ increases, and Lattice Energy \textbf{decreases}.
    \item \textbf{Trend:} $\text{NaCl} > \text{KCl} > \text{RbCl}$.
\end{itemize}

\section*{6. Melting Point: NaCl vs. CCl$_4$}
\begin{itemize}
    \item \textbf{\ce{NaCl}:} Is an ionic compound held together by strong \textbf{electrostatic forces} in a 3D lattice. Breaking these requires massive energy, leading to a high M.P. ($801^\circ\text{C}$).
    \item \textbf{\ce{CCl4}:} Is a non-polar covalent molecule held by weak \textbf{London dispersion forces}. These are easily overcome, resulting in a low M.P. ($-23^\circ\text{C}$).
\end{itemize}

\section*{7. Coordinate Covalent Bond}
A coordinate bond (dative bond) occurs when \textbf{both} electrons in the shared pair come from a single atom.
\textbf{Example:} Ammonium ion (\ce{NH4+}). Ammonia (\ce{NH3}) donates its lone pair to an $H^+$ ion.
\[ \ce{H3N: + H+ -> [H3N \rightarrow H]+} \]

\section*{8. Complex Compounds and Bonding}
A \textbf{coordinate bond} is the primary bond in coordination complexes where ligands donate electron pairs to a central metal ion.
\begin{itemize}
    \item \textbf{\ce{[Cu(NH3)6]Cl2}}: Coordinate bonds between $\ce{Cu^{2+}}$ and $\ce{NH3}$; Ionic bonds between the complex cation and \ce{Cl-}.
    \item \textbf{\ce{K2[Ni(CN)4]}}: Ionic bonds between \ce{K+} and the complex anion; Coordinate bonds between $\ce{Ni^{2+}}$ and \ce{CN-}.
\end{itemize}

\section*{9. Polar vs. Non-polar Covalent Bonds}
\begin{itemize}
    \item \textbf{Polar:} Formed between atoms with different electronegativities (e.g., \ce{H-Cl}). Electrons are shifted toward the more electronegative atom, creating partial charges ($\delta+, \delta-$).
    \item \textbf{Non-polar:} Formed between atoms with similar electronegativity (e.g., \ce{Cl-Cl}, \ce{H-H}). Electrons are shared equally.
\end{itemize}

\section*{10. Bond Length Trend and NH$_4$Cl Bonds}
\textbf{Bond Length Trend:} As s-character increases, the bond length decreases.
\[ sp^3 (25\% s) > sp^2 (33\% s) > sp (50\% s) \]
Therefore, $sp^3$ has the longest bond and $sp$ the shortest.

\textbf{Bonds in \ce{NH4Cl}:} It contains three types of bonds:
1. \textbf{Covalent bonds} (N-H in \ce{NH3}).
2. \textbf{Coordinate bond} (between \ce{NH3} and \ce{H+}).
3. \textbf{Ionic bond} (between \ce{NH4+} and \ce{Cl-}).

\section*{11. Hydrogen Bonding: Water vs. Dichloromethane}
\textbf{Hydrogen Bond:} A strong intermolecular attraction between a Hydrogen atom (bonded to N, O, or F) and a lone pair on another N, O, or F.
\begin{itemize}
    \item \ce{H2O} forms extensive \textbf{Hydrogen bonds}, which require significant energy to break.
    \item \ce{CH2Cl2} only has dipole-dipole interactions, which are much weaker. Thus, water has a much higher boiling point.
\end{itemize}

\section*{12. Solubility: Ethanol vs. Ethane}
\begin{itemize}
    \item \textbf{Ethanol (\ce{C2H5OH}):} Contains an $-OH$ group capable of forming \textbf{Hydrogen bonds} with water molecules.
    \item \textbf{Ethane (\ce{C2H6}):} Is a non-polar hydrocarbon that cannot form H-bonds. It follows the "like dissolves like" rule and remains insoluble in polar water.
\end{itemize}

\section*{13. Bond Formation: CaF$_2$ and PF$_3$}
\begin{itemize}
    \item \textbf{\ce{CaF2} (Ionic):} Calcium (metal) loses 2 electrons to form \ce{Ca^{2+}}. Two Fluorine atoms (non-metals) each gain 1 electron to form \ce{F-}.
    \item \textbf{\ce{PF3} (Covalent):} Phosphorus shares three of its valence electrons with three Fluorine atoms to complete their octets.
\end{itemize}
\textbf{Key Differences:} \ce{CaF2} has a high melting point and conducts electricity when molten, whereas \ce{PF3} has a low boiling point and is a non-conductor.

\section*{14. Lewis Structures}
Below are the Lewis structures for the specified molecules and ions. To represent these in \LaTeX, we use the valence shell electron pair repulsion (VSEPR) logic and formal charge distribution.

\begin{itemize}
    \item \textbf{\ce{NF3}}: Nitrogen is the central atom with one lone pair and three single bonds to Fluorine atoms.
    \item \textbf{\ce{CO3^{2-}}}: Carbon is central, double-bonded to one Oxygen and single-bonded to two Oxygens (which carry the negative charges).
    \item \textbf{\ce{SO4^{2-}}}: Sulfur is central, typically shown with two $S=O$ double bonds and two $S-O^-$ single bonds to minimize formal charge.
    \item \textbf{\ce{XeF4}}: Xenon is central with four single bonds to Fluorine and two lone pairs (square planar geometry).
    \item \textbf{\ce{SO3^{2-}}}: Sulfur is central with one lone pair, one $S=O$ double bond, and two $S-O^-$ single bonds.
    \item \textbf{\ce{SO2Cl2}}: Sulfur is central, double-bonded to two Oxygens and single-bonded to two Chlorines.
    \item \textbf{\ce{XeO3}}: Xenon is central with one lone pair and three $Xe=O$ double bonds (trigonal pyramidal).
\end{itemize}

\section*{15. Formal Charges for Carbonate and Nitrite}
The formula for Formal Charge (FC) is: 
$FC = V - (N + \frac{B}{2})$, where $V$ is valence electrons, $N$ is non-bonding electrons, and $B$ is bonding electrons.

\textbf{Carbonate Ion (\ce{CO3^{2-}}):}
\begin{itemize}
    \item \textbf{Carbon:} $4 - (0 + \frac{8}{2}) = 0$
    \item \textbf{Double-bonded Oxygen:} $6 - (4 + \frac{4}{2}) = 0$
    \item \textbf{Single-bonded Oxygen:} $6 - (6 + \frac{2}{2}) = -1$
\end{itemize}

\textbf{Nitrite Ion (\ce{NO2^{-}}):}
Structure: $[\text{O}=\text{N}-\text{O}]^-$
\begin{itemize}
    \item \textbf{Nitrogen:} $5 - (2 + \frac{6}{2}) = 0$
    \item \textbf{Double-bonded Oxygen:} $6 - (4 + \frac{4}{2}) = 0$
    \item \textbf{Single-bonded Oxygen:} $6 - (6 + \frac{2}{2}) = -1$
\end{itemize}

\section*{16. Resonance Structures of \ce{N2O} and \ce{ClO3-}}
\textbf{Nitrous Oxide (\ce{N2O}):}
Three main resonance contributors:
\begin{enumerate}
    \item $\ce{:{N} \equiv \text{N}^+ - \text{O}:^{3-}}$ (Major contributor: Negative charge on most electronegative atom, Oxygen).
    \item $\ce{^-\text{:N} = \text{N}^+ = \text{O:}}$ (Significant contributor).
    \item $\ce{^{2-}\text{:N} - \text{N}^+ \equiv \text{O}^+}$ (Least important: Large charge separation).
\end{enumerate}

\textbf{Chlorate Ion (\ce{ClO3-}):}
Resonance involves moving the double bond position between the Chlorine and the three Oxygen atoms to distribute the charge.

\section*{17. Polar and Nonpolar Covalent Bonds}
\begin{itemize}
    \item \textbf{Nonpolar Covalent Bond:} Occurs when electrons are shared equally between atoms with identical or very similar electronegativity.
    \begin{itemize}
        \item \textit{Example:} $\ce{Cl-Cl}$ in $\ce{Cl2}$ or $\ce{H-H}$ in $\ce{H2}$.
    \end{itemize}
    \item \textbf{Polar Covalent Bond:} Occurs when one atom is more electronegative than the other, causing an unequal sharing of electrons and a dipole moment.
    \begin{itemize}
        \item \textit{Example:} $\ce{H-Cl}$ (Chlorine is more electronegative) or $\ce{O-H}$ in water.
    \end{itemize}
\end{itemize}

\section*{18. Chemical Reactions of \ce{NaCl} and \ce{AlCl3} with \ce{H2O}}
\begin{itemize}
    \item \textbf{Sodium Chloride (\ce{NaCl}):} Dissolves physically in water; it undergoes hydration but \textbf{not} a chemical reaction (hydrolysis). The pH remains neutral (~7).
    \[ \ce{NaCl(s) ->[H2O] Na+(aq) + Cl-(aq)} \]
    
    \item \textbf{Aluminum Chloride (\ce{AlCl3}):} Undergoes \textbf{hydrolysis} because $\ce{Al^{3+}}$ is a small, highly charged cation. It reacts with water to form an acidic solution.
    \[ \ce{AlCl3 + 3H2O -> Al(OH)3(s) + 3HCl(aq)} \]
    Or more accurately in complex form:
    \[ \ce{[Al(H2O)6]^{3+} + H2O <=> [Al(H2O)5(OH)]^{2+} + H3O+} \]
\end{itemize}

\section*{19. Color Variation in Silver Halides (\ce{AgF}, \ce{AgCl}, \ce{AgBr}, \ce{AgI})}
The variation in color is explained by \textbf{Fajan's Rule} and the phenomenon of \textbf{polarization}.
\begin{itemize}
    \item As we move from \ce{F-} to \ce{I-}, the size of the anion increases.
    \item Larger anions are more easily polarized by the \ce{Ag+} cation. 
    \item Increased polarization leads to increased covalent character and a decrease in the energy gap between the valence and conduction bands.
    \item \ce{AgF} (ionic) is colorless, while \ce{AgI} (most covalent) absorbs blue light and appears yellow.
\end{itemize}

\section*{20. Solubility: Ethanol vs. Ethane}
\begin{itemize}
    \item \textbf{Ethanol (\ce{CH3CH2OH}):} Contains a polar hydroxyl ($-OH$) group that can form \textbf{hydrogen bonds} with water molecules. Hence, it is highly soluble.
    \item \textbf{Ethane (\ce{CH3CH3}):} A non-polar hydrocarbon that only possesses weak London dispersion forces. It cannot form hydrogen bonds with water and is therefore insoluble.
\end{itemize}

\section*{21. Chemical Reactions of \ce{KCl} and \ce{BCl3} with \ce{H2O}}
\begin{itemize}
    \item \textbf{\ce{KCl}:} Being a salt of a strong acid and a strong base, it dissolves in water without undergoing hydrolysis.
    \[ \ce{KCl(s) ->[H2O] K+(aq) + Cl-(aq)} \]
    \item \textbf{\ce{BCl3}:} A covalent halide that undergoes vigorous hydrolysis to form boric acid and hydrochloric acid.
    \[ \ce{BCl3 + 3H2O -> H3BO3 + 3HCl} \]
\end{itemize}

\section*{22. pH, pOH, and Ion Concentrations}
\textbf{Definitions:}
\begin{itemize}
    \item \textbf{pH:} The negative logarithm of the hydrogen ion concentration: $pH = -\log[\ce{H3O+}]$.
    \item \textbf{pOH:} The negative logarithm of the hydroxide ion concentration: $pOH = -\log[\ce{OH-}]$.
    \item \textbf{pH Scale:} Ranges from 0 (acidic) to 14 (basic), with 7 being neutral at $25^\circ\text{C}$.
\end{itemize}

\textbf{Calculations for pOH = 8.5:}
\begin{enumerate}
    \item $[\ce{OH-}] = 10^{-pOH} = 10^{-8.5} = 3.16 \times 10^{-9} \text{ M}$
    \item Since $pH + pOH = 14$: $pH = 14 - 8.5 = 5.5$
    \item $[\ce{H3O+}] = 10^{-pH} = 10^{-5.5} = 3.16 \times 10^{-6} \text{ M}$
\end{enumerate}

\section*{23. Intermolecular vs. Intramolecular Hydrogen Bonding}
\begin{itemize}
    \item \textbf{Intermolecular H-bonding:} Occurs \textbf{between} two or more separate molecules (e.g., \ce{H2O}, Ethanol). It increases boiling points.
    \item \textbf{Intramolecular H-bonding:} Occurs \textbf{within} the same molecule between different functional groups (e.g., o-nitrophenol). It often decreases solubility in water.
\end{itemize}

\section*{24. Predicting Bond Polarity}
Polarity depends on the electronegativity difference ($\Delta\chi$):
\begin{itemize}
    \item \textbf{C--O:} $|2.5 - 3.5| = 1.0$
    \item \textbf{C--S:} $|2.5 - 2.5| = 0.0$
    \item \textbf{H--Br:} $|2.1 - 2.8| = 0.7$
\end{itemize}
\textbf{Conclusion:} The \textbf{C--O bond} is the most polar because it has the largest electronegativity difference.

\section*{25. Polarity of \ce{CCl4} and Bent \ce{CO2}}
\begin{itemize}
    \item \textbf{\ce{CCl4} (Non-polar):} Even though \ce{C-Cl} bonds are polar, the symmetrical tetrahedral geometry causes the individual bond dipoles to cancel each other out.
    \item \textbf{Bent \ce{CO2} (Polar):} In a hypothetical bent geometry, the bond dipoles would not cancel out, resulting in a net molecular dipole moment.
\end{itemize}

\section*{26. Solubility of Ionic Compounds in Water}
Although water is covalent, it is highly polar. 
\begin{itemize}
    \item \textbf{Ion-Dipole Interaction:} The partial negative oxygen of \ce{H2O} attracts the cation ($\ce{Na+}$), and the partial positive hydrogens attract the anion ($\ce{Cl-}$).
    \item If the \textbf{Hydration Energy} released exceeds the \textbf{Lattice Energy} of the crystal, the compound dissolves.
\end{itemize}

\section*{27. Water as a Universal Solvent}
Water's effectiveness as a solvent is due to:
\begin{enumerate}
    \item \textbf{High Dielectric Constant:} It reduces the electrostatic force between ions.
    \item \textbf{Structure:} A bent V-shape (approx. $104.5^\circ$) with $sp^3$ hybridization.
    \item \textbf{Bonding:} Highly polar \ce{O-H} bonds and the ability to form four hydrogen bonds per molecule.
\end{enumerate}

\section*{28. Bonding in Specific Compounds}
Different types of chemical bonds are present in the following:
\begin{itemize}
    \item \textbf{(i) \ce{MgCl2}:} Primarily \textbf{ionic bonds} between the \ce{Mg^{2+}} cation and \ce{Cl-} anions.
    \item \textbf{(ii) \ce{(H2O)_n}:} Individual molecules have polar \textbf{covalent bonds}; however, the "n" signifies an assembly held by \textbf{hydrogen bonds}.
    \item \textbf{(iii) \ce{[Fe(NH3)6]Cl2}:} Contains \textbf{coordinate covalent bonds} between \ce{Fe^{2+}} and \ce{NH3} ligands, and \textbf{ionic bonds} between the complex cation and \ce{Cl-} ions.
\end{itemize}

\section*{29. Acid-Base Concepts}
\begin{enumerate}
    \item \textbf{Arrhenius Concept:} Acids produce \ce{H+} ions in water (e.g., \ce{HCl}), and bases produce \ce{OH-} ions (e.g., \ce{NaOH}).
    \item \textbf{Brønsted-Lowry Concept:} Acids are proton (\ce{H+}) donors, and bases are proton acceptors.
    \item \textbf{Lewis Concept:} Acids are electron-pair acceptors (e.g., \ce{BF3}), and bases are electron-pair donors (e.g., \ce{:NH3}).
\end{enumerate}

\section*{30. Conjugate Acid-Base Pairs}
A conjugate acid-base pair differs by only one proton (\ce{H+}).
\begin{itemize}
    \item \textbf{Conjugate Acids (add \ce{H+}):}
    \begin{itemize}
        \item \ce{HCO3-} $\rightarrow$ \ce{H2CO3}
        \item \ce{HO-} $\rightarrow$ \ce{H2O}
        \item \ce{CN-} $\rightarrow$ \ce{HCN}
        \item \ce{H2PO4-} $\rightarrow$ \ce{H3PO4}
    \end{itemize}
    \item \textbf{Conjugate Bases (remove \ce{H+}):}
    \begin{itemize}
        \item \ce{H2O} $\rightarrow$ \ce{OH-}
        \item \ce{HClO4} $\rightarrow$ \ce{ClO4-}
        \item \ce{HCN} $\rightarrow$ \ce{CN-}
    \end{itemize}
\end{itemize}

\section*{31. pH and pOH Calculations}
\textbf{pH Definition:} $pH = -\log[\ce{H3O+}]$. \textbf{pOH Definition:} $pOH = -\log[\ce{OH-}]$. 
Relation: $pH + pOH = 14$ at $25^\circ\text{C}$.

\begin{itemize}
    \item \textbf{If pH = 5.50:} $pOH = 14 - 5.50 = 8.50$. Thus, $[\ce{OH-}] = 10^{-8.50} = 3.16 \times 10^{-9} \text{ M}$.
    \item \textbf{0.10 M \ce{NaOH}:} Since it is a strong base, $[\ce{OH-}] = 0.10 \text{ M}$. $pOH = -\log(0.10) = 1$. $pH = 14 - 1 = 13$.
    \item \textbf{Human Blood (pH 7.45):} $[\ce{H3O+}] = 10^{-7.45} = 3.55 \times 10^{-8} \text{ M}$. $pOH = 14 - 7.45 = 6.55 \implies [\ce{OH-}] = 2.82 \times 10^{-7} \text{ M}$.
\end{itemize}

\section*{32. Buffer Solutions}
\textbf{Definition:} A solution that resists changes in pH when small amounts of acid or base are added.
\begin{itemize}
    \item \textbf{Acidic Buffer:} Weak acid + its salt with a strong base (e.g., \ce{CH3COOH} + \ce{CH3COONa}).
    \item \textbf{Basic Buffer:} Weak base + its salt with a strong acid (e.g., \ce{NH4OH} + \ce{NH4Cl}).
\end{itemize}
\textbf{Buffer Action:} In an acetate buffer, added \ce{H+} is neutralized by \ce{CH3COO-}, and added \ce{OH-} is neutralized by \ce{CH3COOH}, keeping the pH stable.

\section*{33. Salt Hydrolysis and Solution pH}
Predicting if a solution is acidic, basic, or neutral:
\begin{itemize}
    \item \textbf{\ce{LiClO4}:} Neutral (Salt of strong acid \ce{HClO4} and strong base \ce{LiOH}).
    \item \textbf{\ce{FeCl3}:} Acidic (Small, highly charged $\ce{Fe^{3+}}$ undergoes hydrolysis to release \ce{H+}).
    \item \textbf{\ce{Na3PO4}:} Basic (Anion \ce{PO4^{3-}} comes from a weak acid \ce{H3PO4} and reacts with water to produce \ce{OH-}).
    \item \textbf{\ce{NH4CN}:} Basic (Since $K_b$ of \ce{CN-} is greater than $K_a$ of \ce{NH4+}).
\end{itemize}

\section*{34. Distinguishing Chemicals as Acids/Bases}
Using Lewis/Brønsted definitions:
\begin{itemize}
    \item \textbf{Acids:} \ce{BF3} (Lewis acid), \ce{Ag+} (Lewis acid), \ce{SO3} (Lewis acid).
    \item \textbf{Bases:} \ce{F-} (Lewis base), \ce{:NH3} (Lewis base), \ce{CaO} (Basic oxide).
\end{itemize}

\section*{35. Advanced pH and pOH Calculations}

\textbf{9. pH of 0.1 M \ce{NH3} Solution ($K_b = 1.8 \times 10^{-5}$):}
For a weak base:
\[ [\ce{OH-}] = \sqrt{K_b \cdot C} = \sqrt{(1.8 \times 10^{-5})(0.1)} = 1.34 \times 10^{-3} \text{ M} \]
\[ pOH = -\log(1.34 \times 10^{-3}) = 2.87 \]
\[ pH = 14 - 2.87 = 11.13 \]

\textbf{10. pH of 0.1 M \ce{CH3COOH} Solution ($K_a = 1.8 \times 10^{-5}$):}
For a weak acid:
\[ [\ce{H+}] = \sqrt{K_a \cdot C} = \sqrt{(1.8 \times 10^{-5})(0.1)} = 1.34 \times 10^{-3} \text{ M} \]
\[ pH = -\log(1.34 \times 10^{-3}) = 2.87 \]

\textbf{11. pH of 0.002 M \ce{CH3COOH} (2.3\% ionized):}
\[ [\ce{H+}] = \text{Concentration} \times \text{degree of ionization} = 0.002 \times 0.023 = 4.6 \times 10^{-5} \text{ M} \]
\[ pH = -\log(4.6 \times 10^{-5}) = 4.34 \]

\section*{36. Buffer Action and Mechanism}

\textbf{15. Buffer Action of \ce{NH4OH} and \ce{NH4Cl}:}
This is a basic buffer.
\begin{itemize}
    \item \textbf{Addition of acid (\ce{H+}):} The added \ce{H+} ions react with the weak base \ce{NH4OH} (or \ce{NH3}) to form \ce{NH4+}.
    \[ \ce{NH3(aq) + H+(aq) -> NH4+(aq)} \]
    \item \textbf{Addition of base (\ce{OH-}):} The added \ce{OH-} ions react with the \ce{NH4+} ions from the salt.
    \[ \ce{NH4+(aq) + OH-(aq) -> NH3(aq) + H2O(l)} \]
\end{itemize}

\textbf{16. Buffer Action of \ce{CH3COOH} and \ce{CH3COONa}:}
This is an acidic buffer.
\begin{itemize}
    \item \textbf{Addition of acid (\ce{H+}):} \ce{CH3COO- + H+ -> CH3COOH}
    \item \textbf{Addition of base (\ce{OH-}):} \ce{CH3COOH + OH- -> CH3COO- + H2O}
\end{itemize}

\section*{37. Biological Buffer Systems}
Human blood is a primary example of a biological buffer system. It maintains a pH of approximately 7.4 using:
\begin{enumerate}
    \item \textbf{Bicarbonate Buffer:} \ce{CO2 / HCO3-} system.
    \item \textbf{Phosphate Buffer:} \ce{H2PO4- / HPO4^{2-}} system (mostly intracellular).
    \item \textbf{Protein Buffers:} Hemoglobin and plasma proteins.
\end{enumerate}

\section*{38. Comparison of Lewis and Brønsted-Lowry Concepts}
\begin{itemize}
    \item \textbf{Brønsted-Lowry:} Focuses on \textbf{proton (\ce{H+}) transfer}. An acid is a proton donor, and a base is a proton acceptor.
    \item \textbf{Lewis:} Focuses on \textbf{electron pair transfer}. An acid is an electron pair acceptor, and a base is an electron pair donor.
    \item \textbf{Key Difference:} The Lewis theory is broader; it includes substances like \ce{BF3} and \ce{AlCl3} as acids even though they contain no hydrogen to donate.
\end{itemize}

\section*{39. Conjugate Pairs in Specific Reactions}
In the following reactions, identify the acid (A), base (B), conjugate acid (CA), and conjugate base (CB):

\textbf{(i) \ce{NH3 + H2O <=> NH4+ + OH-}}
\begin{itemize}
    \item \ce{NH3}: Base (B)
    \item \ce{H2O}: Acid (A)
    \item \ce{NH4+}: Conjugate Acid (CA)
    \item \ce{OH-}: Conjugate Base (CB)
\end{itemize}

\textbf{(ii) \ce{HNO3 + H2O <=> H3O+ + NO3-}}
\begin{itemize}
    \item \ce{HNO3}: Acid (A)
    \item \ce{H2O}: Base (B)
    \item \ce{H3O+}: Conjugate Acid (CA)
    \item \ce{NO3-}: Conjugate Base (CB)
\end{itemize}

\section*{40. Introduction to Chemical Kinetics}

\textbf{1. Rate Law and Rate Equations:}
The \textbf{Rate Law} is an expression that relates the rate of a chemical reaction to the molar concentration of the reactants raised to some power.
For a general reaction $aA + bB \rightarrow \text{Products}$, the rate is given by: $Rate = k[A]^x [B]^y$.

\textbf{Rate Equations for Specific Reactions:}
\begin{enumerate}
    \item \ce{H2 + Cl2 -> 2HCl}: $Rate = k[\ce{H2}][\ce{Cl2}]$ (assuming elementary)
    \item \ce{PCl5 -> PCl3 + Cl2}: $Rate = k[\ce{PCl5}]$
    \item \ce{H2 + I2 -> 2HI}: $Rate = k[\ce{H2}][\ce{I2}]$
    \item \ce{2NO + 2H2 -> N2 + 2H2O}: $Rate = k[\ce{NO}]^2[\ce{H2}]$
\end{enumerate}

\section*{41. First-Order Reaction Calculations}

\textbf{2. Rate Constant from Half-life:}
Given $t_{1/2} = 20 \text{ minutes}$. For a first-order reaction:
\[ k = \frac{0.693}{t_{1/2}} = \frac{0.693}{20} = 0.03465 \text{ min}^{-1} \]

\textbf{3. Derivation for First-Order Rate Constant ($A \rightarrow \text{Products}$):}
The differential rate law is $-\frac{d[A]}{dt} = k[A]$. Rearranging and integrating from $t=0$ to $t=t$:
\[ \int_{[A]_0}^{[A]} \frac{d[A]}{[A]} = -k \int_{0}^{t} dt \implies \ln\left(\frac{[A]}{[A]_0}\right) = -kt \]
\[ k = \frac{2.303}{t} \log\left(\frac{[A]_0}{[A]}\right) \]

\textbf{4. Half-life of First-Order Reactions:}
Half-life ($t_{1/2}$) is the time required for the concentration of a reactant to decrease to half its initial value.
Substituting $[A] = \frac{[A]_0}{2}$ into the first-order equation:
\[ t_{1/2} = \frac{\ln(2)}{k} = \frac{0.693}{k} \]
This shows that $t_{1/2}$ is \textbf{independent} of the initial concentration $[A]_0$.

\section*{42. Reaction Progress and Activation Energy}

\textbf{5. Activation Energy ($E_a$):}
Activation energy is the minimum amount of energy required by reactant molecules to undergo a chemical reaction.

\textbf{6. Time to complete 90\% of a reaction:}
Given $50\%$ completion in 23 min ($t_{1/2} = 23$).
\[ k = \frac{0.693}{23} = 0.03013 \text{ min}^{-1} \]
To complete 90\%, $[A] = 0.10[A]_0$:
\[ t_{90\%} = \frac{2.303}{k} \log\left(\frac{1}{0.1}\right) = \frac{2.303}{0.03013}(1) = 76.43 \text{ minutes} \]

\section*{43. Distinction and Kinetics Problems}

\textbf{7. Order vs. Molecularity:}
\begin{itemize}
    \item \textbf{Order:} Sum of powers of concentration terms in the rate law; determined experimentally.
    \item \textbf{Molecularity:} Number of reacting species taking part in an elementary reaction; always a whole number.
\end{itemize}

\textbf{8. Cyclopropane conversion ($k = 6.7 \times 10^{-4} \text{ s}^{-1}$):}
\begin{itemize}
    \item \textbf{(a)} $[A]$ after 8.8 min ($528 \text{ s}$):
    $\ln[A] = \ln(0.25) - (6.7 \times 10^{-4} \times 528) \implies [A] = 0.175 \text{ M}$.
    \item \textbf{(b)} Time to decrease to 0.15 M:
    $t = \frac{2.303}{k} \log\left(\frac{0.25}{0.15}\right) = \frac{2.303}{6.7 \times 10^{-4}}(0.2218) = 762 \text{ s} \approx 12.7 \text{ min}$.
\end{itemize}

\textbf{9. Second-Order Kinetics (Iodine atoms):}
For second order: $\frac{1}{[A]} - \frac{1}{[A]_0} = kt$
Given $k = 7.0 \times 10^9 \text{ /M}\cdot\text{s}$, $[A]_0 = 0.086 \text{ M}$, $t = 120 \text{ s}$.
The concentration $[A]$ after 2 min will be extremely low, essentially zero, due to the very high rate constant.

\section*{44. Lewis Structures and Formal Charges}

\textbf{26. Lewis Structures:}
\begin{itemize}
    \item \textbf{\ce{NF3}:} Nitrogen is the central atom with one lone pair and three single bonds to Fluorine atoms (each F has 3 lone pairs).
    \item \textbf{\ce{CO3^{2-}}}: Carbon is central, double-bonded to one Oxygen and single-bonded to two Oxygens (which carry the negative charges).
    \item \textbf{\ce{SO4^{2-}}}: Sulfur is central, typically shown with two $S=O$ double bonds and two $S-O^-$ single bonds to minimize formal charge.
    \item \textbf{\ce{XeF4}}: Xenon is central with four single bonds to Fluorine and two lone pairs (square planar geometry).
    \item \textbf{\ce{SO3^{2-}}}: Sulfur is central with one lone pair, one $S=O$ double bond, and two $S-O^-$ single bonds.
    \item \textbf{\ce{SO2Cl2}}: Sulfur is central, double-bonded to two Oxygens and single-bonded to two Chlorines.
    \item \textbf{\ce{XeO3}}: Xenon is central with one lone pair and three $Xe=O$ double bonds (trigonal pyramidal).
\end{itemize}

\textbf{27. Formal Charges for Carbonate (\ce{CO3^{2-}}) and Nitrite (\ce{NO2^{-}}):}
Formula: $FC = V - (N + \frac{B}{2})$
\begin{itemize}
    \item \textbf{Carbonate:} Carbon (0), double-bonded Oxygen (0), single-bonded Oxygens ($-1$).
    \item \textbf{Nitrite ($[\ce{O=N-O}]^-$):} Nitrogen (0), double-bonded Oxygen (0), single-bonded Oxygen ($-1$).
\end{itemize}

\section*{45. Solubility and Polar Characteristics}

\textbf{28. Solubility of Ethanol vs. Ethane:}
\begin{itemize}
    \item \textbf{Ethanol (\ce{CH3CH2OH}):} Soluble because its polar $-OH$ group forms \textbf{hydrogen bonds} with water.
    \item \textbf{Ethane (\ce{C2H6}):} Insoluble because it is a non-polar hydrocarbon and cannot form H-bonds.
\end{itemize}

\textbf{Dipole Moments in \ce{CCl4} and Bent \ce{CO2}:}
\begin{itemize}
    \item \textbf{\ce{CCl4}:} Non-polar; the tetrahedral symmetry causes individual bond dipoles to cancel.     \item \textbf{Bent \ce{CO2}:} Polar; in a bent configuration, the vector sum of bond dipoles would not be zero. \end{itemize}

\section*{46. Kinetics: Rate Laws and Order}

\textbf{Integrated Rate Law for First-Order Reactions ($A \rightarrow$ Products):}
\[ \ln[A] = -kt + \ln[A]_0 \quad \text{or} \quad k = \frac{2.303}{t} \log \frac{[A]_0}{[A]} \]

\textbf{Calculations:}
\begin{enumerate}
    \item \textbf{1st order half-life $k=49 \text{ s}^{-1}$:} $t_{1/2} = \frac{0.693}{49} = 0.0141 \text{ s}$.
    \item \textbf{Complete 75\% if $t_{1/2}=2.5 \text{ min}$:} $75\%$ completion equals two half-lives. $t = 2 \times 2.5 = 5.0 \text{ minutes}$.
    \item \textbf{Time for 1/3 remaining ($t_{1/2}=1000 \text{ s}$):}
    \[ k = \frac{0.693}{1000} = 6.93 \times 10^{-4} \text{ s}^{-1} \]
    \[ t = \frac{\ln(3)}{k} = \frac{1.0986}{6.93 \times 10^{-4}} = 1585 \text{ seconds} \]
\end{enumerate}

\section*{47. Enthalpy and Pseudo-First Order Reactions}

\textbf{12. Pseudo-First Order Reaction:}
A reaction that is truly higher-order but behaves as first-order because one reactant is in large excess.
\textbf{Example:} Hydrolysis of cane sugar.
\[ \ce{C12H22O11 + H2O -> C6H12O6 + C6H12O6} \]
Since water is in excess, its concentration remains constant, and $Rate = k'[\ce{C12H22O11}]$.

\textbf{13. Enthalpy of Reaction:}
The heat change at constant pressure during a chemical reaction ($\Delta H = H_{products} - H_{reactants}$).
\begin{itemize}
    \item \textbf{Exothermic:} Energy is released, $\Delta H < 0$.     \item \textbf{Endothermic:} Energy is absorbed, $\Delta H > 0$. \end{itemize}

\textbf{Rate Constant calculation for \ce{2N2O5 -> 4NO2 + O2}:}
90\% reacts in 3600 s ($[A] = 0.1[A]_0$).
\[ k = \frac{2.303}{3600} \log(10) = \frac{2.303}{3600}(1) = 6.4 \times 10^{-4} \text{ s}^{-1} \]

\end{document}