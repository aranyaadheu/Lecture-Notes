\documentclass[a4paper,12pt]{article}
\usepackage[utf8]{inputenc}
\usepackage{geometry}
\geometry{top=1in, bottom=1in, left=1in, right=1in}

% --- Font Configuration (Times New Roman) ---
\usepackage{mathptmx} 

\usepackage{listings}
\usepackage{xcolor}
\usepackage{tcolorbox}
\usepackage{titlesec}

% --- Formatting Configuration ---

% Code Style
\definecolor{codegreen}{rgb}{0,0.6,0}
\definecolor{codegray}{rgb}{0.5,0.5,0.5}
\definecolor{codepurple}{rgb}{0.58,0,0.82}
\definecolor{backcolour}{rgb}{0.96,0.96,0.96}

\lstdefinestyle{CStyle}{
    backgroundcolor=\color{backcolour},   
    commentstyle=\color{codegreen},
    keywordstyle=\color{blue},
    numberstyle=\tiny\color{codegray},
    stringstyle=\color{codepurple},
    basicstyle=\ttfamily\footnotesize, % Monospaced font for code remains cleaner
    breakatwhitespace=false,         
    breaklines=true,                 
    captionpos=b,                    
    keepspaces=true,                 
    numbers=left,                    
    numbersep=5pt,                  
    showspaces=false,                
    showstringspaces=false,
    showtabs=false,                  
    tabsize=4,
    language=C
}

\lstset{style=CStyle}

% Output Box Style
\newtcolorbox{outputbox}{
    colback=black!85,
    colframe=black!85,
    coltext=white,
    boxsep=5pt,
    arc=4pt,
    title=\textbf{Output:},
    coltitle=white,
    fonttitle=\bfseries
}

% Title Page Info
\title{\textbf{Structured Programming Lab Final Assignment} }
\author{\textbf{Name:} Aniruddha Aranya A Dheu\\ \textbf{ID:} 0182520012101113 \\ \textbf{Course:} CSE 1211 - Structured Programming}
\date{04 December, 2025}

\begin{document}

\maketitle
\tableofcontents
\newpage

% ==========================================
% PART 1: MID-TERM TOPICS
% ==========================================

\section{Intro to C}

\subsection{Problem 1: Personal Information}
\textbf{Objective:} Write a program to print your Name, ID, and Department on separate lines.

\begin{lstlisting}
#include <stdio.h>

int main() {
    printf("Name: aranyaadheu\n");
    printf("ID: 23-12345-1\n");
    printf("Dept: Computer Science\n");
    return 0;
}
\end{lstlisting}

\begin{outputbox}
Name: aranyaadheu \\
ID: 23-12345-1 \\
Dept: Computer Science
\end{outputbox}

\subsection{Problem 2: Hello World with Tabs}
\textbf{Objective:} Demonstrate the usage of tab escape sequences.

\begin{lstlisting}
#include <stdio.h>

int main() {
    printf("Hello\tWorld\tfrom\tC!");
    return 0;
}
\end{lstlisting}

\begin{outputbox}
Hello\quad World\quad from\quad C!
\end{outputbox}

\section{Data Types and Format Specifiers}

\subsection{Problem 1: Size of Data Types}
\textbf{Objective:} Display the memory size of `int`, `float`, and `double`.

\begin{lstlisting}
#include <stdio.h>

int main() {
    printf("Size of int: %zu bytes\n", sizeof(int));
    printf("Size of float: %zu bytes\n", sizeof(float));
    printf("Size of double: %zu bytes\n", sizeof(double));
    return 0;
}
\end{lstlisting}

\begin{outputbox}
Size of int: 4 bytes \\
Size of float: 4 bytes \\
Size of double: 8 bytes
\end{outputbox}

\subsection{Problem 2: Floating Point Precision}
\textbf{Objective:} Print a float value with 2 decimal places.

\begin{lstlisting}
#include <stdio.h>

int main() {
    float pi = 3.1415926535;
    printf("Original: %f\n", pi);
    printf("Formatted: %.2f\n", pi);
    return 0;
}
\end{lstlisting}

\begin{outputbox}
Original: 3.141593 \\
Formatted: 3.14
\end{outputbox}

\section{Tokens, Constants, and Operators}

\subsection{Problem 1: Area of Circle}
\textbf{Objective:} Calculate area using a defined constant.

\begin{lstlisting}
#include <stdio.h>
#define PI 3.1416

int main() {
    float r = 5.0, area;
    area = PI * r * r;
    printf("Area of circle with radius %.1f is %.2f", r, area);
    return 0;
}
\end{lstlisting}

\begin{outputbox}
Area of circle with radius 5.0 is 78.54
\end{outputbox}

\subsection{Problem 2: Arithmetic Operators}
\textbf{Objective:} Perform modulus and division operations.

\begin{lstlisting}
#include <stdio.h>

int main() {
    int a = 17, b = 5;
    printf("Quotient: %d\n", a / b);
    printf("Remainder: %d\n", a % b);
    return 0;
}
\end{lstlisting}

\begin{outputbox}
Quotient: 3 \\
Remainder: 2
\end{outputbox}

\section{Errors, Escape Sequences, ASCII, Booleans}

\subsection{Problem 1: Character to ASCII}
\textbf{Objective:} Find the ASCII value of a user input character.

\begin{lstlisting}
#include <stdio.h>

int main() {
    char ch = 'A';
    printf("ASCII value of %c is %d", ch, ch);
    return 0;
}
\end{lstlisting}

\begin{outputbox}
ASCII value of A is 65
\end{outputbox}

\subsection{Problem 2: Boolean Logic (Manual)}
\textbf{Objective:} Simulate boolean behavior using integers (1 for true, 0 for false).

\begin{lstlisting}
#include <stdio.h>

int main() {
    int isSunny = 1; // True
    int isRaining = 0; // False
    
    if(isSunny && !isRaining) {
        printf("Go for a walk.");
    } else {
        printf("Stay inside.");
    }
    return 0;
}
\end{lstlisting}

\begin{outputbox}
Go for a walk.
\end{outputbox}

\section{Control Structures and For Loop}

\subsection{Problem 1: Even or Odd}
\textbf{Objective:} Check if a number is even or odd using `if-else`.

\begin{lstlisting}
#include <stdio.h>

int main() {
    int n = 7;
    if(n % 2 == 0) printf("%d is Even", n);
    else printf("%d is Odd", n);
    return 0;
}
\end{lstlisting}

\begin{outputbox}
7 is Odd
\end{outputbox}

\subsection{Problem 2: Sum of Natural Numbers}
\textbf{Objective:} Sum first 10 numbers using `for` loop.

\begin{lstlisting}
#include <stdio.h>

int main() {
    int sum = 0;
    for(int i = 1; i <= 10; i++) {
        sum += i;
    }
    printf("Sum of 1-10: %d", sum);
    return 0;
}
\end{lstlisting}

\begin{outputbox}
Sum of 1-10: 55
\end{outputbox}

\section{Break, Continue, While, Switch}

\subsection{Problem 1: Skip Number 5 (Continue)}
\textbf{Objective:} Print 1 to 7 but skip 5 using `continue`.

\begin{lstlisting}
#include <stdio.h>

int main() {
    int i = 0;
    while(i < 7) {
        i++;
        if(i == 5) continue;
        printf("%d ", i);
    }
    return 0;
}
\end{lstlisting}

\begin{outputbox}
1 2 3 4 6 7
\end{outputbox}

\subsection{Problem 2: Grade Checker (Switch)}
\textbf{Objective:} Print remarks based on grade letter.

\begin{lstlisting}
#include <stdio.h>

int main() {
    char grade = 'B';
    switch(grade) {
        case 'A': printf("Excellent"); break;
        case 'B': printf("Good"); break;
        default: printf("Fail");
    }
    return 0;
}
\end{lstlisting}

\begin{outputbox}
Good
\end{outputbox}

\section{Loop Practice Problems}

\subsection{Problem 1: Count Digits}
\textbf{Objective:} Count digits in an integer using `while`.

\begin{lstlisting}
#include <stdio.h>

int main() {
    int n = 4521, count = 0;
    while(n != 0) {
        n /= 10;
        count++;
    }
    printf("Total digits: %d", count);
    return 0;
}
\end{lstlisting}

\begin{outputbox}
Total digits: 4
\end{outputbox}

\subsection{Problem 2: Input Validation (Do-While)}
\textbf{Objective:} Keep asking for a positive number.

\begin{lstlisting}
#include <stdio.h>

int main() {
    int n;
    // Hardcoded input simulation
    n = 10; 
    do {
        if (n > 0) printf("Accepted: %d", n);
    } while(n <= 0);
    return 0;
}
\end{lstlisting}

\begin{outputbox}
Accepted: 10
\end{outputbox}

\section{Patterns}

\subsection{Problem 1: Right Angle Triangle}
\textbf{Objective:} Print a triangle of stars.

\begin{lstlisting}
#include <stdio.h>

int main() {
    for(int i=1; i<=3; i++) {
        for(int j=1; j<=i; j++) printf("* ");
        printf("\n");
    }
    return 0;
}
\end{lstlisting}

\begin{outputbox}
* \\
* * \\
* * * \end{outputbox}

\subsection{Problem 2: Square Pattern}
\textbf{Objective:} Print a $3 \times 3$ square of numbers.

\begin{lstlisting}
#include <stdio.h>

int main() {
    for(int i=1; i<=3; i++) {
        for(int j=1; j<=3; j++) printf("%d ", j);
        printf("\n");
    }
    return 0;
}
\end{lstlisting}

\begin{outputbox}
1 2 3 \\
1 2 3 \\
1 2 3 
\end{outputbox}

\section{1D Arrays}

\subsection{Problem 1: Array Sum}
\textbf{Objective:} Calculate the sum of array elements.

\begin{lstlisting}
#include <stdio.h>

int main() {
    int arr[] = {10, 20, 30, 40};
    int sum = 0;
    for(int i=0; i<4; i++) sum += arr[i];
    printf("Sum: %d", sum);
    return 0;
}
\end{lstlisting}

\begin{outputbox}
Sum: 100
\end{outputbox}

\subsection{Problem 2: Find Maximum}
\textbf{Objective:} Find largest number in an array.

\begin{lstlisting}
#include <stdio.h>

int main() {
    int arr[] = {5, 12, 3, 9};
    int max = arr[0];
    for(int i=1; i<4; i++) {
        if(arr[i] > max) max = arr[i];
    }
    printf("Max: %d", max);
    return 0;
}
\end{lstlisting}

\begin{outputbox}
Max: 12
\end{outputbox}

% ==========================================
% PART 2: FINAL EXAM TOPICS
% ==========================================

\newpage
\section{Final: 2D Arrays}

\subsection{Problem 1: Matrix Addition}
\textbf{Objective:} Add two $2 \times 2$ matrices.

\begin{lstlisting}
#include <stdio.h>

int main() {
    int a[2][2] = {{1,2},{3,4}};
    int b[2][2] = {{5,6},{7,8}};
    printf("Sum Matrix:\n");
    for(int i=0; i<2; i++) {
        for(int j=0; j<2; j++) {
            printf("%d ", a[i][j] + b[i][j]);
        }
        printf("\n");
    }
    return 0;
}
\end{lstlisting}

\begin{outputbox}
Sum Matrix:\\
6 8 \\
10 12 
\end{outputbox}

\subsection{Problem 2: Transpose Matrix}
\textbf{Objective:} Print rows as columns.

\begin{lstlisting}
#include <stdio.h>

int main() {
    int mat[2][2] = {{1, 2}, {3, 4}};
    printf("Transpose:\n");
    for(int i=0; i<2; i++) {
        for(int j=0; j<2; j++) {
            printf("%d ", mat[j][i]);
        }
        printf("\n");
    }
    return 0;
}
\end{lstlisting}

\begin{outputbox}
Transpose:\\
1 3 \\
2 4 
\end{outputbox}

\section{Functions and Recursion}

\subsection{Problem 1: Check Prime Function}
\textbf{Objective:} Function that returns 1 if Prime, 0 if not.

\begin{lstlisting}
#include <stdio.h>

int checkPrime(int n) {
    if (n < 2) return 0;
    for(int i=2; i<n; i++) {
        if(n % i == 0) return 0;
    }
    return 1;
}

int main() {
    if(checkPrime(7)) printf("7 is Prime");
    else printf("7 is not Prime");
    return 0;
}
\end{lstlisting}

\begin{outputbox}
7 is Prime
\end{outputbox}

\subsection{Problem 2: Recursive Factorial}
\textbf{Objective:} Find factorial using recursion.

\begin{lstlisting}
#include <stdio.h>

int fact(int n) {
    if (n == 0) return 1;
    return n * fact(n - 1);
}

int main() {
    printf("Factorial of 5: %d", fact(5));
    return 0;
}
\end{lstlisting}

\begin{outputbox}
Factorial of 5: 120
\end{outputbox}

\section{Pointers and DMA}

\subsection{Problem 1: Swap with Pointers}
\textbf{Objective:} Swap two variables using Call by Reference.

\begin{lstlisting}
#include <stdio.h>

void swap(int *x, int *y) {
    int temp = *x;
    *x = *y;
    *y = temp;
}

int main() {
    int a = 10, b = 20;
    swap(&a, &b);
    printf("a: %d, b: %d", a, b);
    return 0;
}
\end{lstlisting}

\begin{outputbox}
a: 20, b: 10
\end{outputbox}

\subsection{Problem 2: Dynamic Memory Allocation}
\textbf{Objective:} Create an array using `malloc`.

\begin{lstlisting}
#include <stdio.h>
#include <stdlib.h>

int main() {
    int *ptr = (int*)malloc(3 * sizeof(int));
    if(ptr == NULL) return 1;
    
    ptr[0] = 10; ptr[1] = 20; ptr[2] = 30;
    
    printf("Dynamic Value: %d", ptr[1]);
    free(ptr);
    return 0;
}
\end{lstlisting}

\begin{outputbox}
Dynamic Value: 20
\end{outputbox}

\section{Strings}

\subsection{Problem 1: String Length}
\textbf{Objective:} Calculate length without `strlen`.

\begin{lstlisting}
#include <stdio.h>

int main() {
    char str[] = "Hello";
    int i;
    for(i=0; str[i] != '\0'; i++);
    printf("Length: %d", i);
    return 0;
}
\end{lstlisting}

\begin{outputbox}
Length: 5
\end{outputbox}

\subsection{Problem 2: Reverse String}
\textbf{Objective:} Reverse a string in place.

\begin{lstlisting}
#include <stdio.h>
#include <string.h>

int main() {
    char str[] = "Code";
    int len = strlen(str);
    for(int i=0; i<len/2; i++) {
        char temp = str[i];
        str[i] = str[len-1-i];
        str[len-1-i] = temp;
    }
    printf("Reversed: %s", str); 
    return 0;
}
\end{lstlisting}

\begin{outputbox}
Reversed: edoC
\end{outputbox}

\section{Structure}

\subsection{Problem 1: Student Data}
\textbf{Objective:} Store and print student details.

\begin{lstlisting}
#include <stdio.h>

struct Student {
    char name[50];
    int age;
};

int main() {
    struct Student s1 = {"John", 21};
    printf("Student: %s, Age: %d", s1.name, s1.age);
    return 0;
}
\end{lstlisting}

\begin{outputbox}
Student: John, Age: 21
\end{outputbox}

\subsection{Problem 2: Add Distances}
\textbf{Objective:} Add two distances in feet/inches.

\begin{lstlisting}
#include <stdio.h>

struct Distance {
    int feet;
    int inch;
};

int main() {
    struct Distance d1 = {5, 8}, d2 = {3, 5}, sum;
    sum.feet = d1.feet + d2.feet;
    sum.inch = d1.inch + d2.inch;
    
    if(sum.inch >= 12) {
        sum.feet++;
        sum.inch -= 12;
    }
    printf("Total: %d' %d\"", sum.feet, sum.inch);
    return 0;
}
\end{lstlisting}

\begin{outputbox}
Total: 9' 1"
\end{outputbox}

\end{document}